% 01_introduction.tex

Here is an example of a citation \cite{ref:oetiker1995not}. Citations are included in the \texttt{cited\_works.bib} file.

\subsection{MIMO Channel Overview}
\label{sect:mimo_model}

In this work, we consider a MIMO channel with a multiple antennas ($n_B \gg 1$) at the transmitter (gNodeB or gNB) servicing a single (or multiple) user equipment(s) (UE) with a single antenna. The network utilizes orthogonal frequency division multiplexing (OFDM) with $N_f$ subcarriers, the $m$-th downlink and uplink channels at the receiver are given as
\begin{align*}
	y_{d,m} &= \mathbf h_{d,m}^H\mathbf w_{t,m}x_{d,m} + n_{d,m}, \\
	y_{u,m} &= \mathbf w_{r,m}^H\mathbf h_{u,m}x_{u,m} + \mathbf w_{r,m}^H\mathbf n_{u,m}.
\end{align*}
The resulting downlink and uplink channel state information (CSI) matrices are given as
\begin{align*} 
	\bar{\mathbf H}_d &= \begin{bmatrix} \mathbf h_{d,1} & \dots & \mathbf h_{d,N_f}\end{bmatrix}^H \in \mathbb C^{N_f \times N_b}, \\
	\bar{\mathbf H}_u &= \begin{bmatrix} \mathbf h_{u,1} & \dots & \mathbf h_{u,N_f}\end{bmatrix}^H \in \mathbb C^{N_f \times N_b}.
\end{align*}
\begin{table}[]
\centering
\caption{MIMO system parameters and variables considered in this work.}
\label{tab:cost-params}
\begin{tabular}{c|c|l}
\toprule
\textbf{Symbol}   & \textbf{Dimension}          & \textbf{Description} \\ \midrule
$y_{d,m}$ 		  & $\mathbb{C}^{1}$ 			& Received downlink symbol on $m$-th subcarrier  \\ \hline
$\mathbf h_{d,m}$ & $\mathbb{C}^{N_b \times 1}$ & Downlink impulse response on $m$-th subcarrier  \\ \hline
$\mathbf w_{t,m}$ & $\mathbb{C}^{N_b \times 1}$ & Transmitter precoding vector for $m$-th subcarrier  \\ \hline
$x_{d,m}$ 		  & $\mathbb{C}^{1}$ 			& Trasmitted symbol on $m$-th subcarrier  \\ \hline
$n_{d,m}$ 		  & $\mathbb{C}^{1}$ 			& Downlink noise on $m$-th subcarrier  \\ \hline
$y_{u,m}$ 		  & $\mathbb{C}^{1}$ 			& Received uplink symbol on $m$-th subcarrier  \\ \hline
$\mathbf h_{u,m}$ & $\mathbb{C}^{N_b \times 1}$ & Uplink impulse response on $m$-th subcarrier  \\ \hline
$\mathbf w_{r,m}$ & $\mathbb{C}^{N_b \times 1}$ & Received precoding vector for $m$-th subcarrier  \\ \hline
$x_{u,m}$ 		  & $\mathbb{C}^{1}$ 			& Received symbol on $m$-th subcarrier  \\ \hline
$\mathbf n_{u,m}$ & $\mathbb{C}^{1}$ 			& Uplink noise on $m$-th subcarrier  \\ \hline
\end{tabular}
\end{table}
To achieve near-capacity transmission rates, the transmitter needs access to an appropriate estimate of $\bar{\mathbf H}_d$ \cite{ref:goldsmith2003capacity}. In time division duplex (TDD), downlink CSI estimation can be performed by using pilots in uplink frames due to channel reciprocity. In contrast, frequency domain duplex (FDD) does not admit channel reciprocity due to frequency-selective channels, and CSI estimates must be acquired using feedback.

Given their dimensionality, feeding back entire CSI matrices is impractical. Instead, we seek a compressed representation of a sparse transformation. We consider the angular-delay representation of CSI matrices. Denote the unitary DFT (inverse DFT) matrix $\mathbf F \in \mathbb C^{n_f \times n_f}$ ($\mathbf F^H \in \mathbb C^{n_f \times n_f}$), and denote the spatial-frequency CSI matrix as $\bar{\mathbf H}$. The angular-delay domain representation $\mathbf H$ is given as
\begin{align*}
	\mathbf H &= \mathbf F^H \bar{\mathbf H} \mathbf F
\end{align*}

\subsection{Channel Model}
\label{sect:channel_model}

For all CSI tests, we mainly rely on the COST2100 MIMO channel model \cite{ref:liu2012cost2100}. We use two datasets with a single base station (gNB) and a single user equipment (UE) in the following scenarios:
\begin{enumerate}
	\item \textbf{Indoor} channels using a 5.3GHz downlink at
	0.001 m/s UE velocity, served by a
	gNB at center of a $20$m$\times 20$m coverage area.
	\item \textbf{Outdoor} channels using a 300MHz downlink at 0.9 m/s UE velocity served by a gNB at center 
	of a $400$m$\times 400$m coverage area.
\end{enumerate}
In both scenarios, we use the parameters listed in Table~\ref{tab:cost-params}.
\begin{table}[]
\centering
\caption{Parameters used for COST2100 simulations for both Indoor and Outdoor datasets.}
\label{tab:cost-params}
\begin{tabular}{c|c|l}
\toprule
\textbf{Symbol} & \textbf{Value} & \textbf{Description} \\ \midrule
$N_b$ 			& 32			 & Number of antennas at gNB  \\ \hline
$N_f$ 			& 1024			 & Number of subcarriers for OFDM link  \\ \hline
$R_d$ 			& 32			 & Number of delay elements kept after truncation  \\ \hline
$N$ 			& $10^6$		 & Total number of samples per dataset  \\ \hline
$T$ 			& 10		 	 & Number of timeslots  \\ \hline
$\delta$		& 40ms, 80ms	 & Feedback delay interval between consecutive CSI timeslots  \\ \bottomrule
\end{tabular}
\end{table}
